\section{Offline Network Intrusion Detection Tasks}
Unlike image classification or natural language processing,
the labeled datasets in network intrusion detection tasks~\cite{KDDCup, DARPA, UNSW1, UNSW, NSL-KDD} lacks a common feature space.
For example, among the two most completed datasets NSL-KDD\cite{NSL-KDD} and UNSW-NB15~\cite{UNSW},
there are only three easy-to-identify common features: src\_bytes, dst\_bytes and service.
As a result, we consider each dataset are an standalone network intrusion detection task.
Among alternative available datasets,
we choose both NSL-KDD dataset and UNSW-NB15 dataset as our targeted tasks.
We list the dimension and amount of the NIDS training datasets that we used in this paper along with those famous image classification datasets in Table~\ref{Tab:Datasets}.

\subsection{NSL-KDD Dataset}
The NSL-KDD dataset originates from the KDDCup 99 dataset~\cite{KDDCup},
which was used for the third International Knowledge Discovery and Data Mining Tool Competition.
NSL-KDD dataset addresses two issues of the KDDCup 99 dataset.
First, it eliminates the redundant records in the KDDCup 99 that take up
78\% and 75\% of the records in train and test set, respectively.
Second, it samples the dataset such that the fraction of the record from a difficulty level
is inversely proportional to its difficulty.
Both enhancements make NSL-KDD dataset more suitable for
evaluating intrusion detection systems.

The train dataset consists of 125,973 TCP connection records, while the test dataset
consists of 22,544 ones.
A record is defined by 41 features, including 9 basic features of individual
TCP connections, 13 content features within a connection and 9 temporal features computed
within a two-second time window, and 10 other features.
Connections in the train dataset are labeled as either normal or one of the 24 attack
types.
There are additional 14 types of attacks in the test dataset, intentionally designed to
test the classifier's ability to handle novel attacks.
The task of the classifier is to identify whether a connection is normal or one of the
4 categories of attacks, namely denial of service (DoS), remote to local (R2L), user to
root (U2R) and probing, also known as 5-class classification problem.

\subsection{UNSW-NB15 Dataset}
Similar to KDDCup 99 dataset, the UNSW-NB15 dataset is generated by simulating normal
and attack behaviors in a hardware testbed.
The simulation is conducted in the Cyber Range Lab of the Australian Centre for Cyber Security (ACCS)
and 49 features in the dataset is extracted by a chain of software tools also developed by ACCS.
The structure of the features is similar to that of KDDCup 99: 5 flow features,
13 basic features, 8 content features, 9 time features and 12 other features.
However, there are only a couple of common features to the NSL-KDD dataset.
The size of the dataset is 257,673 in term of flow records, 175,341 of which are used for
training set and the rest are for testing.
There are nine types of attacks in the dataset.
The only common type of attack between UNSW-NB15 and NSL-KDD is DoS.
The new attacks in UNSW-NB15 are analysis, backdoor, exploits, fuzzers, generic, reconnaissance, shellcode, and worms.
In this project, we consider the 2-class classification problem for UNSW-NB15 dataset: the
task of the classifiers is to predict a given traffic is either normal or malicious.
