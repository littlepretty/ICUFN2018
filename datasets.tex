\section{Off-line Network Intrusion Detection Tasks}
Unlike image classification or natural language processing,
the labeled datasets in network intrusion detection tasks lacks a common feature space~\cite{KDDCup, DARPA, UNSW, NSL-KDD}.
For example, between the two datasets NSL-KDD\cite{NSL-KDD} and UNSW-NB15~\cite{UNSW},
we only identify five common features, i.e., src\_bytes, dst\_bytes, service, flag and duration.
As a result, we have to consider each dataset as a standalone network intrusion detection task.
We choose NSL-KDD and UNSW-NB15 datasets as our targeted tasks.
% Table~\ref{Tab:Datasets} summarizes both tasks' training datasets
% along with some famous image classification datasets for comparison.

\subsection{NSL-KDD Dataset}
The NSL-KDD dataset originates from the KDDCup 99 dataset~\cite{KDDCup},
but addresses two issues of the KDDCup 99 dataset.
First, it eliminates the redundant records in the KDDCup 99, i.e., 
78\% of the training set and 75\% of the testing set.
Second, it samples the dataset so that the number of records belonging to one difficulty level is inversely proportional to its difficulty.
The changes make the NSL-KDD dataset suitable for evaluating intrusion detection systems.
The training dataset consists of 125,973 TCP connection records, while the testing dataset consists of 22,544 records.
A record is defined by 41 features, including 9 basic features of individual TCP connections, 13 content features within a connection, 9 temporal features computed within a two-second time window, and 10 other features.
Connections in the training dataset are labeled as normal or one of the 24 attack types.
There are additional 14 types of attacks in the testing dataset, intentionally designed to test the classifier's ability to handle unknown attacks.
A classifier identifies whether a connection is normal or belongs to one of the four categories of attacks, namely denial-of-service (DoS), remote-to-local (R2L), user-to-root (U2R), and probing (also known as the 5-class classification problem).

\subsection{UNSW-NB15 Dataset}
Similar to the KDDCup 99 dataset, the UNSW-NB15 dataset is generated by simulating normal and attack behaviors in a hardware testbed.
The simulation is conducted in the Cyber Range Lab of the Australian Centre for Cyber Security (ACCS) and 49 features in the dataset are extracted by a chain of software tools developed by ACCS.
The structure of the features is similar to that of KDDCup 99 including 5 flow features, 13 basic features, 8 content features, 9 time features, and 12 other features.
However, there are only five common features between to the UNSW-NB15 and NSL-KDD datasets.
The dataset has 257,673 flow records, among which 175,341 are used for
training set and the rest are for testing.
There are nine types of attacks in the dataset.
The only type of attack in common between UNSW-NB15 and NSL-KDD is DoS.
The new attacks in UNSW-NB15 are analysis, backdoor, exploits, fuzzers, generic, reconnaissance, shellcode, and worms.
In this paper, we consider the 2-class classification problem for the UNSW-NB15 dataset. The task is to predict whether a given flow record is normal or malicious.

% \begin{table}[]
% \centering
% \caption{Popular Datasets used in Deep Learning v.s. Available Network Intrusion Detection Datasets}
% \label{Tab:Datasets}
% \begin{tabular}{c|c|r|r}
% \multicolumn{1}{c|}{Domain}                          & Dataset       & Training Examples & Feature Dimension \\
% \hline
% \hline
% \multirow{6}{*}{Image}                               & MNIST         & 60,000        & 784     \\
%                                                      & SVHN          & 600,000       & 3072    \\
%                                                      & CIFAR-10      & 60,000        & 3072    \\
%                                                      & ImageNet      & 1.2 million   & 196,608 \\
% \hline
% \multicolumn{1}{c|}{\multirow{2}{*}{NIDS}}           & UNSW-NB15     & 175,341       & 42      \\
% \multicolumn{1}{l|}{}                                & NSL-KDD       & 125,973       & 41      
% \end{tabular}
% \end{table}
